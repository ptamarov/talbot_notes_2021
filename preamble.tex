\documentclass[a4paper, twoside]{article} 

% for print
\usepackage[
	top = 1.15 in, 
	bottom = 1.25 in,
	left = 1.15 in, 
	right = 1.15 in,
	includehead]{geometry}
% font sizes
\usepackage{scrextend}
\changefontsizes{11pt}
 
\usepackage[all]{nowidow}

\usepackage[utf8x]{inputenc}
\usepackage{amsfonts,amssymb,amsmath}
\usepackage{amsrefs}
\usepackage{url}

\BibSpec{article}{%
    +{}  {\PrintAuthors}                {author}
    +{,} { \textit}                     {title}
    +{.} { }                            {part}
    +{:} { \textit}                     {subtitle}
    +{,} { \PrintContributions}         {contribution}
    +{.} { \PrintPartials}              {partial}
    +{,} { }                            {journal}
    +{}  { \textbf}                     {volume}
    +{}  { \PrintDatePV}                {date}
    +{,} { \issuetext}                  {number}
    +{,} { \eprintpages}                {pages}
    +{,} { }                            {status}
    +{,} { \url}                        {url}    % <---- ADDED
    +{,} { \PrintDOI}                   {doi}
    +{,} { available at \eprint}        {eprint}
    +{}  { \parenthesize}               {language}
    +{}  { \PrintTranslation}           {translation}
    +{;} { \PrintReprint}               {reprint}
    +{.} { }                            {note}
    +{.} {}                             {transition}
    +{}  {\SentenceSpace \PrintReviews} {review}
}

\usepackage{graphicx}
\usepackage[
	%spanish.lcroman,
	british]{babel}
\usepackage{caption}
%\usepackage{paragraphs}
\usepackage{mathdots}
\usepackage{mathtools} 
\usepackage{mathbbol}
\usepackage{amsmath}
\usepackage{amsfonts}
\usepackage{amssymb}
\usepackage{booktabs}
%
%\newcommand{\mathsymbol}[2]{
%	\begin{tikzpicture}
%        \node at(0,0){};
%        \node at#1{$#2$};
%    \end{tikzpicture}
%}

%white star
\usepackage{bbding}
\newcommand{\wstar}{\mathbin{\text{\tiny\FiveStarOpen}
}}

\usepackage{xcolor}
% Link colors and styles, change accordingly, these are the official colours of Trinity 
\definecolor{trinityblue}{rgb}{0.05, 0.45,0.75}
\definecolor{trinitygray}{rgb}{0.33, 0.34,0.35}


% front matter stlye 
\renewenvironment{abstract}{%
\small\begin{center}
\begin{minipage}{.9\textwidth}
%\textbf{\textcolor{newcol}{Abstract.}}
}
{\par\noindent\end{minipage}\end{center}\vspace{3 em}}
%
\makeatletter
\renewcommand\@maketitle{%
\hfill
\begin{center}\begin{minipage}{0.9 	\textwidth}
\centering
\vskip 2em
\let\footnote\thanks 
{\LARGE \@title \par }
\vspace{-.5 em}
%\hrulefill
\vskip 1 em
{\large \@author \par}
\vspace{3.5 em}

\end{minipage}\end{center}
\par
}
\makeatother
%
%%%%%%%%%%%%%%%%%%%%%%%%%%%%%%%%%%%%%%%%%%%%%%%%%%%%%%


\usepackage{tikz-cd}
%\usetikzlibrary{calc}
\usetikzlibrary{matrix,arrows}
\usepackage{stackrel}

\usepackage[shortlabels]{enumitem} %nice lists
\usepackage{stmaryrd} %double brackets
\usepackage{setspace} %line spacing
\spacing{1.15}
%\onehalfspacing
%\setlength{\parskip}{.35 em}
\usepackage{etoolbox}
\usetikzlibrary{trees}

% textual claims in equations
\newcommand\claim[2][.8]{%
  \begin{minipage}{#1\displaywidth}%
  \itshape
  #2
  \end{minipage}%
}

% lists
\usepackage{enumitem}
\listfiles
\setlist[enumerate]{label= (\arabic*)}

% smaller sep in enumerate
\newenvironment{tenumerate}{
\begin{enumerate}
  \setlength{\itemsep}{0pt}
  \setlength{\parskip}{0pt}
}{\end{enumerate}}

% smaller sep in itemize
\newenvironment{titemize}{
\begin{itemize}
  \setlength{\itemsep}{0pt}
  \setlength{\parskip}{0pt}
}{\end{itemize}}
\patchcmd{\section}{\normalfont}{\normalfont\large}{}{}

% fourier
%\usepackage{fourier}
%\usepackage[T1]{fontenc}

% nimbus roman
\usepackage{mathptmx}
\usepackage[T1]{fontenc}

% customize fonts 
\usepackage[bb=ams, cal=euler, scr=rsfso , frak=euler]{mathalfa}

\usepackage{wrapfig, framed, caption}

% running color in paper
\definecolor{newcol}{rgb}{0,0,0}
%{0.03, 0.27, 0.49}

% fancy headers setup
\usepackage{fancyhdr}
\pagestyle{fancy}
\fancyhead[RE]{\small\it Little disks operads and
configuration spaces}
\fancyhead[LO]{\small\it Viva Talbot 2021}
\fancyhead[RO,LE]{\small\textbf{\thepage}}
\renewcommand{\headrulewidth}{0 pt}
\cfoot{}

\fancypagestyle{frontpage}{% use whatever name you please
  \fancyhf{}% clear all the headers and footers
  % now define the contents of the various fields
  \fancyfoot[C]{\thepage}%
  % 
}

\usepackage{letltxmacro}

%\AtBeginDocument{%
%    \LetLtxMacro\refa{\ref}%
%    \DeclareRobustCommand{\ref}[2][]{(\refa#1{#2})}%
%}
%

\usepackage[pdftex, colorlinks,bookmarks 
= true,bookmarksnumbered = true]{hyperref}

% let \[ and \] be the same as \begin{equation} and \end{equation}
\makeatletter
\AtBeginDocument{%
  \let\[\@undefined
  
\DeclareRobustCommand{\[}{\begin{equation}}%
  \let\]\@undefined
  
  \DeclareRobustCommand{\]}{\end{equation}}%
}
\makeatother 
% but only print equation numbers if needed
%\mathtoolsset{showonlyrefs,showmanualtags}

\usepackage{stmaryrd}
\usepackage{amsthm}
\usepackage{thmtools}

% theorem style
\newtheoremstyle{mytheorem}
  {\topsep}   % ABOVESPACE
  {\topsep}   % BELOWSPACE
  {\itshape}  % BODYFONT
  {0pt}       % INDENT (empty value is the same as 0pt)
  {\bfseries\color{newcol}} % HEADFONT
  {%\color{newcol}{.}
  }         % HEADPUNCT
  {5pt plus 1pt minus 1pt} % HEADSPACE
  {}          % CUSTOM-HEAD-SPEC

\theoremstyle{mytheorem}
\newtheorem{theorem}{Theorem}[section]
\newtheorem{lemma}[theorem]{Lemma}
\newtheorem{cor}[theorem]{Corollary}
\newtheorem{prop}[theorem]{Proposition}

% definition style
\newtheoremstyle{mydefinition}
  {\topsep}   % ABOVESPACE
  {\topsep}   % BELOWSPACE
  {}  % BODYFONT
  {0pt}       % INDENT (empty value is the same as 0pt)
  {\bfseries\color{newcol}} % HEADFONT
  {\color{newcol}}         % HEADPUNCT
  {5pt plus 1pt minus 1pt} % HEADSPACE
  {}          % CUSTOM-HEAD-SPEC
  
\theoremstyle{mydefinition}
\newtheorem{definition}[theorem]{Definition}
\newtheorem{observation}[theorem]{Observation}
\newtheorem{example}[theorem]{Example}
\newtheorem{remark}[theorem]{Remark}
\newtheorem{problem}[theorem]{Problem}

% named theorem %
% for specifying a name
\theoremstyle{plain} % just in case the style had changed
\newcommand{\thistheoremname}{}
\newtheorem{genericthm}[theorem]{\thistheoremname}
\newenvironment{namedthm}[1]
  {\renewcommand{\thistheoremname}{#1}%
   \begin{genericthm}}
  {\end{genericthm}}

% url colors
\hypersetup{colorlinks,
	linkcolor={trinityblue},
		citecolor={trinityblue},
			urlcolor={trinityblue}}  


\newtheoremstyle{mytheorem2}
  {\topsep}   % ABOVESPACE
  {\topsep}   % BELOWSPACE
  {\itshape}  % BODYFONT
  {0pt}       % INDENT (empty value is the same as 0pt)
  {\bfseries\color{newcol}} % HEADFONT
  {\color{newcol}{.}
  }         % HEADPUNCT
  {5pt plus 1pt minus 1pt} % HEADSPACE
  {}          % CUSTOM-HEAD-SPEC


\theoremstyle{mytheorem2}
\newtheorem*{theorem*}{Theorem}
\newtheorem*{lemma*}{Lemma}
\newtheorem*{corollary*}{Corollary}
\newtheorem*{conjecture*}{Conjecture}
\newtheorem*{question*}{Question}

\DeclarePairedDelimiter\abs{\lvert}{\rvert}

% command macros
%\renewcommand{\qedsymbol}{$\blacktriangleleft$}

% generic stuff
\newcommand\place{\mathord-}
\newcommand{\imor}{\interleave\kern-.45em\longrightarrow}
\newcommand{\vv}{\vert}
\newcommand{\Conf}{\operatorname{Conf}}
\newcommand{\Map}{\operatorname{Map}}
\newcommand{\Emb}{\operatorname{Emb}}
\newcommand{\Imm}{\operatorname{Imm}}
\newcommand{\dd}{\partial}
\newcommand{\?}{\,?\,}
\newcommand{\NN}{\mathbb N}
\renewcommand{\k}{[k]}
\newcommand{\n}{[n]}
\newcommand{\kk}{\Bbbk}

% special fonts for categories
\newcommand\cat[1]{\mathsf{#1}}
\newcommand\spe[1]{\mathbf{#1}}
\newcommand\bicom[1]{{{}^{#1}\category{Mod}\mkern1mu{}^{#1}}}
\newcommand{\kMod}{{}_\kk\cat{Mod}}
\newcommand{\kmod}{{}_\kk\cat{mod}}
\newcommand{\kCh}{{}_\kk\cat{Ch}}
\newcommand{\Sp}{\cat{Sp}}
\newcommand{\kSp}{{}_\kk\cat{Sp}}

% operads stuff
\newcommand\oper[1]{\mathsf{#1}}
\newcommand{\antishriek}{\text{\raisebox{\depth}{\textexclamdown}}}
\newcommand\id{\mathrm{id}}
\newcommand{\Def}{\mathsf{Def}}
\newcommand{\Der}{\operatorname{Der}}
\newcommand{\Coder}{\operatorname{Coder}}
\newcommand\Cob[1]{\Omega{#1}}
\newcommand\A{\mathsf{As}}
\newcommand\C{\mathsf{Com}}
 

%% SMALL HYPHEN %%
\mathchardef\hy="2D % Define a "math hyphen"
\newcommand\Alg{\hy\mathsf{Alg}}

\newcommand\F{\mathcal{F}}
\renewcommand\P{\mathcal{P}}
\newcommand{\Op}{\mathsf{Op}}
\newcommand{\OO}{\mathcal{O}}
\newcommand\QQ{\mathcal{Q}}
\newcommand\PL{\mathsf{PreLie}_{\NN}}
\newcommand\PA{\mathsf{Perm}_{\NN}}
%homological algebra
\newcommand{\Cxs}{\mathsf{Ch}}
\newcommand{\Tot}{\operatorname{Tot}}
\newcommand{\Ad}{\operatorname{Ad}}
\newcommand{\g}{\mathfrak g}
\newcommand{\coker}{\operatorname{coker}}
\newcommand{\HH}{\mathrm{HH}}
\newcommand{\gr}{\mathrm{gr}\,}
\newcommand{\HC}{\mathrm{HC}}
\renewcommand{\H}{\mathrm{H}}
\newcommand{\Sq}{\operatorname{Sq}}
\newcommand{\Ho}{\operatorname{Ho}}
\renewcommand{\tt}{\otimes}
\newcommand{\Ext}{\operatorname{Ext}}
\newcommand{\Tor}{\operatorname{Tor}}
\newcommand{\End}{\operatorname{End}}
\newcommand\hoc[1]{\operatorname{hocof}{#1}}

\usepackage[new]{old-arrows} %% allows long hook arrow

% new definition font
\DeclareTextFontCommand{\new}{\color{newcol}\em}


% author information 
\newcommand{\Addresses}{{% additional braces for segregating \footnotesize
  \bigskip
  \footnotesize

  \textsc{22 Inselstrasse, Leipzig, Germany}\par\nopagebreak
  \textit{E-mail address:} \texttt{tamaroff@mis.mpg.de}
  }}
  
 
\usepackage{sectsty}
\chapterfont{\color{newcol}}  % sets colour of chapters
\sectionfont{\color{newcol}}  % sets colour of sections
\subsectionfont{\color{newcol}}  % sets colour of sections
 
%%%% Table of Contents Style %%%%%
 
% style for titles of sections
\usepackage{titletoc}
\titlecontents{chapter}
[0.2em] %
{\bigskip}
%{\contentslabel[\thecontentslabel.]{2em}\hspace{0.667em}}%\thecontentslabel
{\makebox[2em][r]{\thecontentslabel.}\hspace{0.333em}}%\thecontentslabel
{\hspace*{-2em}}
{\hfill\contentspage}[\smallskip]

\titlecontents{section}% <section>
[0.2em]% <left>
{\small}% <above-code>
{\thecontentslabel.\hspace{3pt}}%<numbered-entry-format>; you could also 
%use  {\thecontentslabel. } to show the numbers
{}% <numberless-entry-format>
{\enspace\titlerule*[0.5pc]{.}\contentspage}%<filler-page-format>
\titlecontents*{subsection}% <section>
[1 em]% <left>
{\footnotesize}% <above-code>
{\thecontentslabel. \hspace{3pt}}% <numbered-entry-format>; you could also 
%use {\thecontentslabel. } to show the numbers
{}% <numberless-entry-format>
{}% <filler-page-format>
[ --- \ ]% <separator>
[]% <end>
\setcounter{tocdepth}{2}% Display up to \subsection in ToC


\setlength\parindent{0 em}
\setlength\parskip{3 pt}

\raggedbottom 
%\makeindex
