\include{PREAMBLE}

\title{\setstretch{0.9} %will give nice line sep
	\textbf{Little disks operads and configuration spaces}}
\author{Notes by Pedro Tamaroff}
\date{}

\begin{document}
\maketitle

\begin{abstract}
\textbf{Speaker's abstract.} Operads are objects that govern categories of algebras. Initially introduced in the sixties to study iterated loop spaces, they have proved useful in several areas of mathematics. In most of these applications, the little disks operads play a central role. In the first part of this talk, we will focus on one of the applications studied in the 2015 Talbot Workshop, Goodwillie--Weiss embedding calculus, which will serve as an ``excuse'' to introduce operads. In the second part of this talk, I will set out some of the recent developments regarding the links between the little disks operads and the real homotopy types of configuration spaces of manifolds. (Second part based on joint works with Ricardo Campos, Julien Ducoulombier, Pascal Lambrechts, and Thomas Willwacher.)	
\end{abstract}
\bigskip
\tableofcontents

\thispagestyle{frontpage}
%\pagebreak

\section{The embedding calculus}
 
 The motivation to construct the
 embedding calculus on manifolds is 
 the computation of the homotopy type of
 the space $\Emb(M,N)$ of 
 all embeddings $f : M\longrightarrow N$ of one
 manifold into another. Recall that $f$ is
 an \emph{immersion} if its derivative
 $T_pf : T_p M\longrightarrow T_q N$ is 
 injective at each $p\in M$, and it is a (smooth)
 \emph{embedding} if it is an immersion and a 
 topological embedding.
 
 The computation of such space $\Emb(M,N)$ of
 embeddings is highly non-trivial: if $M=S^1$ and 
 $N=S^3$, then the $0$th homotopy group
 $\pi_0\Emb(S^1,S^3)$ consists of all isotopy
 classes of smooth knots, and determining and
 studying these already spans a whole area of
 mathematics, knot theory. 
 
 The fact the previous example is slightly involved
 stems from the fact that the codimension of 
 $S^1$ in $S^3$ is two: as soon as $M$ has
 codimension at least three in $N$, then $\Emb(M,N)$
 is a path connected space. For example, one can
 unknot any embedding $S^1\longrightarrow S^4$.
 Nonetheless, higher homotopy and homology groups
 of $\Emb(M,N)$ are non-trivial, and give interesting
 and fine invariants of $M$ and $N$ that depend on 
 their smooth structure of them, and not only on
 their homotopy type. 
 
 The immediate problem one encounters when trying
 to compute with the functor $\Emb(-,N)$ is that it
 not continuous (in the categorical sense): if
 $M$ is a union of submanifolds $V\cup U$ then
 $\Emb(M,N)$ is \emph{not} equal to the pullback
 of the cospan
 
 \[
 \begin{tikzcd}
  \Emb(V,N) &  \arrow[r]  \Emb(U\cap V,N)
 \arrow[l]       &  \Emb(U, N).
 \end{tikzcd}
 	 	  	\]
 	 	  	
 Concretely, if one is able to build embeddings
 $V\hookrightarrow N$ and $U\hookrightarrow N$
 that coincide in $V\cap U$,
 it is not always possible to extend them to an
 embedding defined on $M=V\cup U$: the resulting
 map will be an immersion, but may fail to be
 injective on $U\cap V$. 
 
 The idea of Goodwillie--Weiss calculus is to 
 \emph{approximate} the functor $F :M\longmapsto
 \Emb(M,N)$ by a tower of functors
 \[\label{eq:tower} \cdots 
 \longrightarrow F_4 	
 \longrightarrow F_3 	
 \longrightarrow F_2 	
 	\longrightarrow F_1 	
 		\longrightarrow F_0   \] 
 under $F$ that are, in a precise sense, polynomial functors.
 As a first step, one can consider the space of 
 \emph{immersions} $M\longrightarrow N$, which is
 polynomial of order one or, what is the same,
 linear, in the sense that $\Imm(M,N)$ is
 the limit of the span
  \[
 \begin{tikzcd}
  \Imm(V,N) &  \arrow[r]  \Imm(U\cap V,N)
 \arrow[l]       &  \Imm(U, N).
 \end{tikzcd}
 	 	  	\]
 To define what it means for a functor to be ``of
 order at most $k$'' for some $k\in \NN$ requires
 a bit more care, an involves the combinatorics of
 cubical posets. We point the reader to~\cite{Calculus}
 for details. Informally, the cornerstone of the
 Goodwillie--Weiss calculus is the following 
 result:
 
 \begin{theorem*}
 Let $F(V) = \Emb(V,N)$ and suppose that $M$ has codimension
 at least three in $N$. There exists a tower
 of functors over $F$ as in~\eqref{eq:tower} such that the
 map
 \[ F(M) \longrightarrow \operatorname{holim}_k F_k(M)\]
  restricts to a homotopy equivalence of the base
  point components. \textcolor{trinityblue}{Moreover,
  the (homotopy type of the?) functors $F_k(M)$ can be 
  computed in some way using configurations of $k$ points
  of $M$ in $N$.}\qed
   \end{theorem*}
 
  We will in fact take a slightly different approach,
  and use little disks operads and configuration spaces of
  points to obtain a description of the homotopy type of
  $\Emb(M,N)$.
  
 \section{Configuration spaces of points}
  
  Not the original work of Goodwillie--Weiss but the
 result of a refinement of their work over the years. 
  Span of maybe two decades. 
  
  We now aim to approximate $\Emb(M,N)$ using
  \emph{configurations spaces}. For each $r\in \NN$,
  we define the \emph{ordered configuration space of
  $r$ points in $M$} by
  \[ \Conf_M(r) = 
  	\{ (x_1,\ldots,x_r)\in M^r 
  	: \text{ $x_i\neq x_j$ for all $i\neq j$} \}. 
  \]
  They appeared initially in the study of braids and
  braid groups by [Name?], and later in the work
  of Fadell--Neuwirth and V. I. Arnol'd.  Note that,
  more or less by definition, one has
  that the braid group $B_r$ is equal to 
  $\pi_1(\Conf_{D^2}(r))$  
  
 The sequence of spaces $(\Conf_M(r))_{r\geqslant 0}$
 forms a \emph{symmetric sequence} in spaces, that is, 
 it is a sequence of spaces with corresponding actions
 of the symmetric groups, and we will write it $\Conf_M$. 
 They allow us to embed $\Emb(M,N)$ into
 \[ \Map_\Sigma(\Conf_M,\Conf_N) = 
 	\prod_{r\geqslant 0}
 		\hom_{\Sigma_r}(\Conf_M(r),\Conf_N(r))\] 
by assigning an embedding $f:M\longrightarrow N$
to the map $\Phi_f$ that assigns a 
configuration $(x_1,\ldots,x_r)$
in $M$ to the configuration $(f(x_1),\ldots,f(x_r))$. 
There are corresponding constructions
for unframed manifolds. The maps $\Phi_f$ coming from embeddings $f:M\longrightarrow N$
enjoy some additional compatibility properties:
\begin{tenumerate}
	\item \emph{Forgetting points:} the map $\Phi_f$ commutes
	with the maps induced by the forgetful maps 
	$\pi_i : \Conf_M(r) 
	\longrightarrow \Conf_M(r-1)$ that forget the $i$th point
	in the domain. 
	\item \emph{Continuity:} if a configuration $\vec{x_0}$ is
	close to a configuration $\vec{x_1}$ in $M$, then their
	images under $\Phi_f$ will be close as configurations
	in $N$.
	\end{tenumerate}
	
	We would like to relax these two conditions ``up to 
	homotopy''. But what does this mean?
	
	\section{Operads and their modules}
	
	One can use operads to achieve this. Let us consider
	a richer version of configuration spaces, that give us
	some more wiggle room, by replacing points with disks. 
	For each $r\in \NN$, define
	\[ D_m(r) = \Emb_\square(D^m \sqcup \cdots \sqcup D^m , D^m ) \] 
	the space of rectilinear\footnote{We only allow for dilations and translations} embeddings of $r$ disjoint
	$m$-disks in another $m$-disk. Similarly, define
		\[ D_M(r) = \Emb(D^m \sqcup \cdots \sqcup D^m , M ) \] 
and note that $\Conf_M(r)$ and $D_M(r)$ are homotopy
equivalent spaces. The upshot is that the symmetric sequence
$D_m$ forms a (topological) operad. Namely, it comes equipped
with composition maps
\[ D_m(k)\times D_m(r_1)\times\cdots D_m(r_k) 
	\longrightarrow D_m(r_1+\cdots r_k) \] 
that satisfy certain associativity and equivariance relations.
More briefly, it forms a monoid in the monoidal category of
symmetric sequences under the circle product
\[(X\circ Y)(n) = 
	\bigsqcup_{k\geqslant 0} X(k)\times_{S_k} 
		Y[\lambda], \]
where $Y[\lambda]$ is the $S_k$-module obtained as the
disjoint union of all permutations of $Y(\lambda_1)\times
\cdots \times Y(\lambda_k)$ for $\lambda$ a partition of $n$. With
this at hand, what we want is an equivariant associative map
\[ D_m\circ D_m \longrightarrow D_m. \]
 
 Moreover, the sequence $D_M$ is a right $D_m$-module, in
 the sense there is a map $D_M\circ D_m \longrightarrow D_m$
 that is compatible with the operad structure of $D_m$. 
 \emph{Note:} since $n\geqslant m$, there is an inclusion
 of operads $D_m\longrightarrow D_n$, and in particular
 the right $D_n$-module $D_N$ can be viewed as a right
 $D_m$-module, which we will do in what follows.
	
 We now observe that an embedding $f:M\longrightarrow N$
 produces for us a map $D_M \longrightarrow D_N$ that is
 not just a map of symmetric sequences, but in fact a map
 right $D_m$-modules. In this way, it makes sense to form
 the mapping space $\Map_{D_m}(D_M,D_N)$
 and we obtain a map
 \[ \Emb(M,N) \longrightarrow  \Map_{D_m}(D_M,D_N)\] 
 by virtue of the additional compatibility properties we
 observed before. With this at hand, we can state
 the following theorem:
 
 \begin{theorem*}[Goodwillie--Weiss, Arone--Turchin, Turchin, Boavida--Weiss, Sinha, ...] If $M$ has codimension at least
 three in $N$, then the map above induces a homotopy
 equivalence
 \[ \Emb(M,N) \longrightarrow \mathbb R \Map_{D_m}(D_M,D_N)\]
 where the right hand side is the \emph{derived}\footnote{For 
 details, the reader can consult the work of B. Fresse~\cite{}.} 
 mapping space of $D_m$-module maps $D_M\longrightarrow D_N$.
 \qed
  \end{theorem*}
 
 It is useful to note that the derived mapping space describes
 ``$F_\infty$ term'' in the Goodwillie--Weiss tower for
 $\Emb(M,N)$, and that one can use truncated versions of
 mapping spaces to describe the finite stages of the tower.
 
 The upshot of this result is that if we can understand
 the homotopy type of the configuration spaces as right
 modules over the little disk operad, then we can understand
 the homotopy type of the space of embeddings. However,
 there is a 
 trade-off: we now need to determine the homotopy type of
 a very simple manifold into another ---a disjoint union of
 points--- but nonetheless the functor 
 $M\longmapsto \Conf_M(r)$ is not
 easy to compute. In particular, it is not homotopy
 invariant! For example, the point has empty higher
 configuration spaces, but the contractible space
 $D^2$ has $\Conf_{D^2}(2) \simeq S^1$. 
 
 One may suspect that the problem above is that the
 manifold $D^2$ is not closed (compact), 
 but Longoni--Salvatore
 have proved that the lens spaces $L_{7,1}$ and $L_{7,2}$,
 which are homotopy equivalent, have non-homotopy
 equivalent configuration spaces. However, these 
 are non-simply connected spaces, and the following
 is still an open question:
 
 \begin{question*}
 Is it true that two simply connected closed manifolds
 of the same homotopy type have homotopy equivalent
 configuration spaces?
 \end{question*}
 
 
 \subsection{Bonus: deloopings}
 
 The little disks operads were initially introduced to
 study and identify when a space can be delooped. That is,
 given a space $X$, when is it weakly homotopy equivalent
 to a space of the form
 \[ \Omega^n(Y) = \Map(D^n;S^{n-1}, Y;\ast) \] 
 for some other space $Y$? Any loop space $X=\Omega Y$ is an $H$-space and, as such,
 the monoid $\pi_0(X)$ is in fact group like, in the sense
 it is a group under the induced product of $X$. Moreover,
 it is an algebra over the little $n$-disks operad: there
 are equivariant maps
 $D_n(k) \times X^k \longrightarrow X$
 obtained by inserting maps $D^n\longrightarrow Y$
 into a disk configuration to obtain a new map 
 $D^n\longrightarrow Y$, which is associative in a precise
 sense.\footnote{Which one?}
 
 Conversely, one can consider such a $D_n$-algebra $X$
 which, in particular, comes equipped with a single
 homotopy class of operations $m: X^2\longrightarrow X$
 originating from $\pi_0(D_n(2))$, making $\pi_0(X)$
 into an associative monoid. We say $X$ is a \emph{group-like}
 $D_n$-algebra if this monoid is in fact a group.
 
 With this at
 hand, we can state the following result, going back to work
 of Beck, Boardman-Vogt, May, Segal and Stasheff. The
 theorem, due to J. P. May, provides a useful theorem to
 determine when a space can be delooped.
  
 \begin{theorem*} A space $X$ is weakly equivalent to
 an $n$th loop space $\Omega^n(Y)$ if, and only if,
 it is a group-like $D_n$-algebra.  \qed
 \end{theorem*}
 
\bibliographystyle{alpha}
\bibliography{biblio}


\Addresses

\end{document}
